\section{Conclusão}

O desenvolvimento da segunda fase do projeto \textit{GereZoo} permitiu consolidar conhecimentos avançados de Programação Orientada a Objetos e Engenharia de Software. O presente capítulo sintetiza as principais conclusões retiradas deste trabalho.

A adoção da arquitetura N-Tier revelou-se fundamental para a organização do código. Embora tenha aumentado o número de ficheiros e projetos na solução, a separação clara entre dados, lógica e apresentação facilitou imenso a deteção de erros e a implementação de novas funcionalidades, como a classe \textit{Reptil}, sem necessidade de alterar o código base da aplicação.

A implementação de padrões de desenho, especificamente o \textit{Singleton} e a \textit{Factory}, conferiu ao projeto um nível de profissionalismo superior, resolvendo problemas de gestão de memória e criação de objetos de forma elegante. A persistência de dados garantiu a utilidade real da aplicação.

Conclui-se que os objetivos propostos foram integralmente cumpridos. Como trabalho futuro, sugere-se a substituição da interface de consola por uma interface gráfica (Windows Forms ou WPF), o que seria facilitado pela arquitetura atual, visto que apenas seria necessário substituir a camada \textit{GereZoo}.