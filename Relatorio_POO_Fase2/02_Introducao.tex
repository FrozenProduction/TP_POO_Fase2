\section{Introdução}

O desenvolvimento de software moderno exige não apenas a implementação de funcionalidades, mas também a adoção de arquiteturas que promovam a manutenção, a escalabilidade e a organização do código. O presente capítulo contextualiza o trabalho desenvolvido na Fase 2 do projeto.

\subsection{Motivação e Enquadramento}
No âmbito da Licenciatura em Engenharia de Sistemas Informáticos, a unidade curricular de Programação Orientada a Objetos propõe a criação de um sistema de gestão. A motivação para esta segunda fase reside na necessidade de transformar um protótipo inicial numa aplicação profissional, capaz de armazenar dados de forma persistente e estruturada segundo as boas práticas da indústria.

\subsection{Objetivos}
Os principais objetivos desta fase foram:
\begin{itemize}
    \item Reestruturar o projeto numa arquitetura N-Tier (Várias Camadas).
    \item Implementar a persistência de dados (leitura e escrita em ficheiro).
    \item Aplicar padrões de desenho como \textit{Singleton} e \textit{Factory}.
    \item Utilizar LINQ e Expressões Lambda para consultas a dados.
    \item Desenvolver um mecanismo robusto de tratamento de exceções.
\end{itemize}

\subsection{Estrutura do Documento}
Este relatório encontra-se organizado em cinco capítulos. Após esta introdução, o capítulo 2 apresenta o enquadramento teórico. O capítulo 3 detalha o trabalho desenvolvido e a implementação. O capítulo 4 apresenta a análise dos resultados e, por fim, o capítulo 5 expõe as conclusões.