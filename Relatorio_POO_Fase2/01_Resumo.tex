\section*{Resumo}

O presente relatório descreve a segunda fase do trabalho prático da unidade curricular de Programação Orientada a Objetos, que visa o desenvolvimento de uma aplicação para a gestão de um Jardim Zoológico, denominada \textit{GereZoo}.

Enquanto a primeira fase focou-se na modelação de classes e herança, esta segunda fase teve como objetivo principal a reestruturação da solução para uma arquitetura em camadas (N-Tier), visando uma clara separação de responsabilidades. Foram implementados mecanismos de persistência de dados através de serialização binária, garantindo a continuidade da informação entre execuções. Adicionalmente, aplicaram-se padrões de desenho (\textit{Design Patterns}), nomeadamente \textit{Singleton} e \textit{Factory Method}, e utilizou-se LINQ para a manipulação eficiente de coleções.

Os resultados obtidos demonstram uma aplicação robusta, capaz de gerir animais, tarefas e bilheteira, com tratamento de exceções personalizado e validação de regras de negócio, cumprindo assim todos os requisitos propostos no enunciado.