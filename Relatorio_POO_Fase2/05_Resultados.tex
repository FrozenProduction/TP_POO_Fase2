\section{Análise e Discussão de Resultados}

Este capítulo apresenta a análise dos resultados obtidos com a implementação da segunda fase do projeto, focando-se na validação da arquitetura e das funcionalidades críticas.

\subsection{Validação Funcional da Arquitetura N-Tier}
Os testes realizados durante o desenvolvimento demonstraram que a separação em camadas foi implementada com sucesso. A aplicação de consola (camada de apresentação) interage exclusivamente com a camada de lógica de negócio (\texttt{ZooBL}), que por sua vez valida os dados antes de recorrer à camada de dados (\texttt{ZooDB}).

Esta estrutura provou ser robusta. Por exemplo, ao tentar inserir um animal com um identificador já existente, a camada de regras deteta o conflito via LINQ e lança a exceção personalizada \texttt{AnimalException}, impedindo a corrupção dos dados na memória.

\subsection{Persistência de Dados}
Um dos principais requisitos desta fase era a persistência. Verificou-se que o mecanismo de serialização binária implementado no \textit{Singleton} \texttt{ZooDB} funciona corretamente.
Ao fechar a aplicação, os dados em memória (listas de animais) são serializados para o ficheiro \texttt{dadosZoo.bin}. Ao reiniciar a aplicação, estes dados são desserializados e carregados novamente, garantindo a continuidade do estado do sistema entre execuções.

\subsection{Integração de Padrões}
A utilização do padrão \textit{Factory} na criação de animais simplificou o código na camada de apresentação, abstraindo a complexidade da instanciação das classes concretas (\texttt{Mamifero}, \texttt{Ave}, \texttt{Reptil}).

Adicionalmente, a implementação da classe \texttt{Bilhete} permitiu validar o uso de membros estáticos (\texttt{static}) para contadores globais (total vendido e total arrecadado), demonstrando um conceito distinto do padrão \textit{Singleton} utilizado para a base de dados central.

\subsection{Validação e Testes}
A robustez da aplicação foi validada através de dois vetores:
\begin{enumerate}
    \item \textbf{Testes Funcionais:} Execução da aplicação de consola, onde se verificou a persistência de dados (os animais mantêm-se após reiniciar o programa) e a correta listagem polimórfica das diferentes espécies.
    \item \textbf{Testes Unitários:} Execução da bateria de testes automatizados do projeto \texttt{ZooTests}, que confirmaram que a camada \texttt{ZooBL} está a bloquear corretamente IDs duplicados e dados nulos, garantindo a qualidade do código.
\end{enumerate}