\section{Enquadramento Teórico e Prático}

Este capítulo apresenta os fundamentos teóricos essenciais que suportam as decisões de implementação tomadas durante o desenvolvimento do projeto.

\subsection{Arquitetura N-Tier}
A arquitetura em camadas, ou N-Tier, é um padrão de arquitetura de software que separa a aplicação em camadas lógicas e físicas. Tipicamente, esta separação inclui a Camada de Apresentação (UI), a Camada de Lógica de Negócio (BLL) e a Camada de Acesso a Dados (DAL) \parencite{microsoft_arch}. Esta abordagem permite que a manutenção numa camada não afete necessariamente as outras, promovendo a modularidade.

\subsection{Padrões de Desenho (Design Patterns)}
Os padrões de desenho são soluções típicas para problemas comuns no design de software.
\begin{itemize}
    \item \textbf{Singleton:} Garante que uma classe tenha apenas uma instância e fornece um ponto de acesso global a ela. É fundamental para a gestão centralizada de dados em memória \parencite{gof_patterns}.
    \item \textbf{Factory Method:} Define uma interface para criar um objeto, mas deixa as subclasses decidirem que classe instanciar. Permite que o código seja independente das classes concretas que precisa criar.
\end{itemize}

\subsection{Persistência e Serialização}
A persistência refere-se à característica de um estado que sobrevive ao processo que o criou. Em C\#, a serialização binária permite converter o estado de um objeto num fluxo de bytes para armazenamento em disco, permitindo a sua reconstrução posterior \parencite{ms_docs_serialization}.